
\subsection{Bestimmung der Ionenstärke und $K_{ps}$}

\begin{align*}
	I = \frac{1}{2} \sum c_i z_i^2 \qquad B = \sqrt{\frac{2 e_0^2 N_A}{\epsilon_0 \epsilon_r k_b T}} \qquad  g(I) = \frac{\sqrt{I}}{1 + B r \sqrt{I}} \\
\end{align*}

\begin{table}[H]
	\centering
	\begin{tabular}{llll}
		\toprule
		Temp. [C] & Ionenstärke [mol/$m^3$] & B           & g(l)        \\
		\midrule
		25        & 45                      & 104407920,2 & 4,501326352 \\
		25        & 45                      & 104407920,2 & 4,501326352 \\
		25        & 45                      & 104407920,2 & 4,501326352 \\
		25        & 60                      & 104407920,2 & 4,945965598 \\
		25        & 130                     & 104407920,2 & 6,219240179 \\
		30        & 130                     & 105245280,4 & 6,196650778 \\
		30        & 130                     & 105245280,4 & 6,196650778 \\
		\bottomrule
	\end{tabular}
\end{table}